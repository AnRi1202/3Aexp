\documentclass[a4paper, twocolumn]{article} % A4用紙, 二段組
\usepackage[utf8]{inputenc} % UTF-8エンコーディング
\usepackage{amsmath, amssymb} % 数式パッケージ
\usepackage[dvipdf]{graphicx} % 画像挿入
\usepackage[dvipdfmx,breaklinks=true]{hyperref}
\usepackage{geometry} % レイアウト設定
\usepackage{caption} % キャプション調整
\usepackage{float} % 画像配置の制御
\usepackage{titlesec} % セクションの見た目変更
\usepackage{siunitx} % SI単位系
\usepackage{multicol} % 必要に応じて複数列対応

\begin{document}

% タイトル
\maketitle

\section*{考察事項}
レポートは以下の事項について文献を調査し、実験結果を示しながら記述すること。出典は必ず示すこと。ChatGPT 等を参照する場合も出典は書籍などの文献に依ること。

\subsection*{(1) ネットワークアナライザの動作原理、装置の仕組みを論じよ。}
% ここにネットワークアナライザの動作原理と装置の仕組みについて記述する

\subsection*{(2) 実験課題 (2) の特性測定結果から、各ストリップラインの特性インピーダンス、伝搬定数 (減衰定数と位相定数) を求めよ。}
% ここに実験課題 (2) の特性測定結果と各ストリップラインの特性インピーダンス、伝搬定数について記述する

\subsection*{(3) 実験課題 (6) で自分が設計・試作した平面回路素子に関し、設計指針と理論的な特性予測に関し記述せよ。次に、試作素子において実測した特性との比較考察を行え。予測通りにならなかった場合、その原因を考察せよ。}
% ここに実験課題 (6) で設計・試作した平面回路素子の設計指針と理論的な特性予測、実測した特性との比較考察を記述する

\subsection*{(4) 実験課題 (7) で作製したパッチアンテナについて、その放射パターンの形状がどのように決まるか、考察せよ。}
% ここに実験課題 (7) で作製したパッチアンテナの放射パターンの形状について記述する

\subsection*{(5) パッチアンテナが他の無線機器からの電波を受信するなど、実験期間中に経験した興味深い結果、事象について考察せよ。}
% ここにパッチアンテナが他の無線機器からの電波を受信するなど、実験期間中に経験した興味深い結果、事象について記述する

\end{document}

