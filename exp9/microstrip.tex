\documentclass[a4paper, twocolumn]{article} % A4用紙, 二段組
\usepackage[utf8]{inputenc} % UTF-8エンコーディング
\usepackage{amsmath, amssymb} % 数式パッケージ
\usepackage[dvipdf]{graphicx} % 画像挿入
\usepackage[dvipdfmx,breaklinks=true]{hyperref}
\usepackage{geometry} % レイアウト設定
\usepackage{caption} % キャプション調整
\usepackage{float} % 画像配置の制御
\usepackage{titlesec} % セクションの見た目変更
\usepackage{siunitx} % SI単位系
\usepackage{multicol} % 必要に応じて複数列対応

% ページの余白を調整
\geometry{
  top=25mm,
  bottom=25mm,
  left=20mm,
  right=20mm
}

% タイトルとセクションのフォーマット調整
\titleformat{\section}{\large\bfseries}{\thesection}{1em}{} % セクションタイトル
\titleformat{\subsection}{\normalsize\bfseries}{\thesubsection}{1em}{} % サブセクションタイトル

% タイトル情報
\title{後期実験9(前半) 無線通信を支える技術
~アンテナと通信方式の実践的理解~}
\author{学籍番号: 03240470 \and 氏名: 井手陸大}
\date{\today}

\begin{document}

\maketitle

\section{ネットワークアナライザの動作原理と装置の仕組み}
ネットワークアナライザは, 高周波回路やデバイスの特性を評価するための計測器であり, 特に反射係数や伝送特性を測定する際に使用される. その中でも, ベクトルネットワークアナライザ(VNA)は, 信号の振幅と位相の両方を測定できるため, より詳細な特性評価が可能である.

\subsection{動作原理}
VNAは, 信号源から生成された高周波信号をデバイスアンダーテスト(DUT)に入力し, DUTからの反射信号や透過信号を測定する. これにより, DUTのSパラメータ(散乱パラメータ)を取得し, 反射特性や伝送特性を評価する. Sパラメータは, 入射波と反射波, 透過波の関係を示す複素数で表され, 振幅と位相の情報を含む.

\subsection{装置の仕組み}
VNAは主に以下のコンポーネントで構成される:
\begin{enumerate}
    \item \textbf{信号源}: 広範囲の周波数で安定した高周波信号を生成する.
    \item \textbf{信号分離器(パワースプリッタ)}: 生成された信号を基準信号とDUTへの入射信号に分離する.
    \item \textbf{方向性結合器(カプラ)}: DUTからの反射信号や透過信号を分離して検出する.
    \item \textbf{受信機}: 基準信号とDUTからの信号を受信し, 振幅と位相を測定する.
\end{enumerate}

測定された信号はデジタル処理され, スミスチャートや対数振幅, 位相, 群遅延などの形式で表示される. DUTの特性評価を精度良く行うためには, これらのデータが不可欠である.

\subsection{校正}
高精度な測定を行うため, VNAは測定系自身が持つ誤差成分を補正する校正を行う. 一般的な校正手法として, オープン(開放), ショート(短絡), ロード(無反射終端器)を用いたSOLT法がある. 校正により, 測定系の誤差要因である方向性, ソースマッチ, ロードマッチ, 伝送周波数レスポンス, 反射周波数レスポンス, アイソレーション(リーケージ)を補正し, 高い測定確度を実現する.


\section{実験課題 (2): 特性測定結果}

\subsection{実験条件}
実験では,以下の条件下で特性インピーダンス \( Z_0 \), 位相定数 \(\beta\), および減衰定数 \(\alpha\) を求めた.
基板の厚さ \( h \) を 1 mm, 比誘電率 \( \epsilon_r = 4.7 \), 周波数 \( f = 1 \, \text{GHz} \) とし, マイクロストリップライン幅の比 \( W/h \) および \(\sqrt{\epsilon_{\text{eff}}}\) の組み合わせを次の通りとした:

\[
\begin{array}{|c|c|}
\hline
\sqrt{\epsilon_{\text{eff}}} & W/h \\
\hline
1.85 & 4.34 \\
1.85 & 3.19 \\
1.90 & 2.35 \\
1.90 & 1.89 \\
2.00 & 1.51 \\
\hline
\end{array}
\]

\subsection{特性インピーダンスの計算}
特性インピーダンス \( Z_0 \) は以下の式で計算する:
\[
Z_0 = \frac{377}{W/h + \frac{2}{\pi} \left( 1 + \ln\left(1 + \frac{\pi}{2} W/h\right)\right)}
\]

\subsection{位相定数と減衰定数の計算}

位相定数 \(\beta\) は以下の式を用いて計算する:
\[
\beta = \omega \sqrt{\epsilon \mu_0}, \quad \epsilon = \epsilon_0 \epsilon_{\text{eff}}, \quad \omega = 2 \pi f
\]
ここで, 実効誘電率 \( \epsilon_{\text{eff}} \) はストリップライン幅 \( W \) に基づき次式で計算する:
\[
\epsilon_{\text{eff}} = \frac{\epsilon_r + 1}{2} + \frac{\epsilon_r - 1}{2} \left(1 + 12 \frac{h}{W}\right)^{-1/2}, \quad \epsilon_r = 4.7
\]

減衰定数 \(\alpha\) は以下の式で計算する:
\[
\alpha \approx \frac{1}{2} \sqrt{\epsilon_{\text{eff}}} \beta \tan \delta + \frac{\epsilon_{\text{eff}} R_s}{\zeta h}
\]
ここで, \( \tan \delta = 0.02 \), \( R_s = \sqrt{\frac{\omega \mu_0}{2 \sigma_{\text{cond}}}}, \, \sigma_{\text{cond}} = 5.8 \times 10^7 \, \text{S/m} \), \( h = 1 \, \text{mm} \).

基板の比誘電率が \( \epsilon_r = 4.7 \) であることから,基板は FR-4 と仮定した. また,損失正接 \( \tan\delta \) は FR-4 の典型値である 0.02 を採用した. 導体金属は銅と仮定し,その導電率を \( \sigma_{\text{cond}} = 5.8 \times 10^7 \, \text{S/m} \) とした.

\subsection{計算結果}
実験条件下での計算結果を以下に示す:

\[
\begin{array}{|c|c|c|c|}
\hline
\sqrt{\epsilon_{\text{eff}}} & W/h & \alpha \, (\text{Np/m}) & \beta \, (\text{rad/m}) \\
\hline
1.85 & 4.34 & 281,984 & 1.409 \times 10^7 \\
1.85 & 3.19 & 265,095 & 1.366 \times 10^7 \\
1.90 & 2.35 & 251,540 & 1.331 \times 10^7 \\
1.90 & 1.89 & 243,105 & 1.309 \times 10^7 \\
2.00 & 1.51 & 237,347 & 1.293 \times 10^7 \\
\hline
\end{array}
\]


\end{document}

