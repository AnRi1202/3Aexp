\documentclass[a4paper, twocolumn]{article} % A4用紙, 二段組
\usepackage[utf8]{inputenc} % UTF-8エンコーディング
\usepackage{amsmath, amssymb} % 数式パッケージ
\usepackage[dvipdf]{graphicx} % 画像挿入
\usepackage[dvipdfmx,breaklinks=true]{hyperref}
\usepackage{geometry} % レイアウト設定
\usepackage{caption} % キャプション調整
\usepackage{float} % 画像配置の制御
\usepackage{titlesec} % セクションの見た目変更
\usepackage{siunitx} % SI単位系
\usepackage{multicol} % 必要に応じて複数列対応

% ページの余白を調整
\geometry{
  top=25mm,
  bottom=25mm,
  left=20mm,
  right=20mm
}

% タイトルとセクションのフォーマット調整
\titleformat{\section}{\large\bfseries}{\thesection}{1em}{} % セクションタイトル
\titleformat{\subsection}{\normalsize\bfseries}{\thesubsection}{1em}{} % サブセクションタイトル

% タイトル情報
\title{後期実験9}
\author{学籍番号: 03240470 \and 氏名: 井手陸大}
\date{\today}

\begin{document}

% タイトル
\maketitle

\section{課題1}
\subsection{観測場所}
観測は2号館12階の教室で実施した.

\subsection{観測データの整理}
\begin{itemize}
    \item 周波数: 80 MHz
    \item 強度: -80 dBm
    \item 観測されたスペクトラム:
    \begin{itemize}
        \item 1枚目の画像はFM信号である.
        \item 2枚目の画像はNHK信号である.
    \end{itemize}
\end{itemize}

\subsection{観測スペクトラムの画像}
以下に観測したスペクトラムのスクリーンキャプチャを示す.

\begin{figure}[H]
    \centering
    \includegraphics[width=\linewidth]{data1/day1/FM.png}
    \caption{片側カラム内の画像}
    \label{fig:image1}
\end{figure}

\begin{figure}[H]
    \centering
    \includegraphics[width=0.8\linewidth]{data1/day1/NHK.png}
    \caption{NHK信号のスペクトラム}
    \label{fig:nhk_spectrum}
\end{figure}


\section*{課題2: FM送受信機の実装}

\subsection*{考察事項}
FM変復調器の原理を数式とブロックダイアグラムを用いて説明する。

\subsection*{FM変調の原理}
FM変調は、音声信号 \(m(t)\) を用いて搬送波の周波数を変化させる方式である。送信信号は以下で表される:
\[
s(t) = A \cos \left( 2 \pi f_c t + 2 \pi k \int m(t) \, dt \right)
\]
ここで、
\begin{itemize}
    \item \(f_c\):搬送波の中心周波数
    \item \(k\):周波数変調指数
    \item \(\int m(t) dt\):音声信号の積分
\end{itemize}

USRPでは、複素IQ信号として搬送されるため、以下の形に変換される:
\[
s_{IQ}(t) = e^{j \left( 2 \pi f_c t + 2 \pi k \int m(t) \, dt \right)}
\]

\subsection*{FM復調の原理}
FM復調では、受信信号 \(s(t)\) から瞬時周波数を抽出する。理論上は以下の操作を行う:
\[
f_{\text{inst}}(t) = \frac{1}{2 \pi} \frac{d}{dt} \arg \left( s_{IQ}(t) \right)
\]
これにより、音声信号 \(m(t)\) を復元する。

\subsection*{ブロックダイアグラムとの対応}
以下に、FM送受信機のブロックダイアグラムを示す。


\subsection*{受信についての考察}
\begin{itemize}
    \item \textbf{IQ rateの限界}:
    理論上、標本化定理より帯域幅の2倍まで下げられる。ただし、ノイズやフィルタリングの影響を考慮する必要がある。
    \item \textbf{音質とIQ rateの関係}:
    IQ rateが低いと、帯域幅が狭くなり高周波成分が失われるため、音質が劣化する。
\end{itemize}

\subsection*{送信についての考察}
\begin{itemize}
    \item \textbf{適切な \(k\) の値}:
    実験では \(k \approx 10^{-6}\) 程度が適切だった。この値は音声信号の振幅とFM変調器の設計による。
    \item \textbf{リサンプリングと積分の順序}:
    リサンプリングを積分前に行うことで、高精度な積分が可能になる。
\end{itemize}

\subsection*{FMの音質が良い理由}
FMでは振幅成分が一定で、ノイズが加わっても信号の周波数変化に影響を与えにくいため、AMより音質が良い。また、FMは広い帯域幅を利用できるため、高周波成分を含む信号を忠実に再現できる。



\end{document}
