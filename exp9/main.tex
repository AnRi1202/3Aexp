\documentclass[a4paper, twocolumn]{article} % A4用紙, 二段組
\usepackage[utf8]{inputenc} % UTF-8エンコーディング
\usepackage{amsmath, amssymb} % 数式パッケージ
\usepackage[dvipdf]{graphicx} % 画像挿入
\usepackage[dvipdfmx,breaklinks=true]{hyperref}
\usepackage{geometry} % レイアウト設定
\usepackage{caption} % キャプション調整
\usepackage{float} % 画像配置の制御
\usepackage{titlesec} % セクションの見た目変更
\usepackage{siunitx} % SI単位系
\usepackage{multicol} % 必要に応じて複数列対応

% ページの余白を調整
\geometry{
  top=25mm,
  bottom=25mm,
  left=20mm,
  right=20mm
}

% タイトルとセクションのフォーマット調整
\titleformat{\section}{\large\bfseries}{\thesection}{1em}{} % セクションタイトル
\titleformat{\subsection}{\normalsize\bfseries}{\thesubsection}{1em}{} % サブセクションタイトル

% タイトル情報
\title{後期実験9(後半) 無線通信を支える技術
~アンテナと通信方式の実践的理解~}
\author{学籍番号: 03240470 \and 氏名: 井手陸大}
\date{\today}

\begin{document}

% タイトル
\maketitle

\section{課題1}
\subsection{観測場所}
観測は2号館12階の教室で実施した.

\subsection{受信パラメータ}
FMラジオ(Tokyo FM 80.0MHz)および地上デジタル(26ch, NHK-E, 548MHz-554MHz)の信号を以下の設定で受信した.

\paragraph{FMラジオ (Tokyo FM 80.0MHz)}
\begin{itemize}
    \item デバイス名: 192.168.10.3 または 192.168.10.11
    \item IQレート: 400kHz
    \item キャリア周波数: 80MHz
    \item アクティブアンテナ: RX2
    \item サンプル数: 100,000
\end{itemize}

\paragraph{地上デジタル (26ch, NHK-E, 548MHz-554MHz)}
\begin{itemize}
    \item デバイス名: 192.168.10.3 または 192.168.10.11
    \item IQレート: 10MHz
    \item キャリア周波数: 551MHz
    \item アクティブアンテナ: RX2
    \item サンプル数: 200,000
\end{itemize}

\subsection{観測結果}
観測されたスペクトラムのスクリーンキャプチャを以下に示す.

\begin{figure}[H]
    \centering
    \includegraphics[width=\linewidth]{data1/day1/FM.png}
    \caption{FMラジオのスペクトラム}
    \label{fig:fm_spectrum}
\end{figure}

\begin{figure}[H]
    \centering
    \includegraphics[width=\linewidth]{data1/day1/NHK.png}
    \caption{地上デジタル(NHK)のスペクトラム}
    \label{fig:nhk_spectrum}
\end{figure}

\subsection{考察}
FMラジオの観測結果について, Figure \ref{fig:fm_spectrum} に示されるスペクトラムは, 受信したIQ信号をFFT処理した結果である. 観測周波数帯の中心周波数は80MHzであるが, スペクトルアナライザ上ではキャリア周波数がUSRPによって除去され, ベースバンド表示(中心周波数が0Hz)の形になっている. 従って, 図の中心を基準として, スペクトル全体を80MHzに平行移動すると実際の受信信号のスペクトルに対応する. ナイキスト周波数と標本化定理に基づき, 400kHzのIQレートでは片側200kHzまでの情報を正確に記録できる. この設定はFMラジオの帯域幅(約200kHz)を十分にカバーしている.

一方, 地上デジタル(NHK)の観測結果について, Figure \ref{fig:nhk_spectrum} に示されるスペクトラムは548MHzから554MHzまでの信号をキャリア周波数551MHzを中心として受信したものである. 地上デジタル放送は6MHzの帯域幅を持つOFDM方式であり, Figure \ref{fig:nhk_spectrum} に見られるように全体的に平坦なスペクトル形状を示している. IQレート10MHzは, 地上デジタル放送の帯域幅6MHzをカバーするために必要な最低限のレート(ナイキスト周波数: 6MHz)を十分に満たしており, 受信データが損失なく取得できている.

どちらの観測結果も, キャリア周波数が取り除かれているため, スペクトラムアナライザ上ではベースバンド信号として表示されている. また, Figure \ref{fig:fm_spectrum} では中心付近にピークが見られるが, これは受信信号の低周波成分が大きく寄与していることを示している. FMラジオの音声信号や地上デジタルの信号がキャリア周波数を中心に構成されることを考慮すると, スペクトラムアナライザ上の表示形状は妥当であると考えられる.

\subsection{フーリエ変換の特性を用いた考察}
実際に観測されている波形は, USRPによってキャリア周波数\(f_c\)が取り除かれた後の信号である. このため, スペクトルアナライザで表示される信号は, ベースバンド信号のスペクトルであり, その中心周波数は0Hzとして描かれている. フーリエ変換の周波数シフト性に基づくと, 観測スペクトルをキャリア周波数\(f_c\)だけ平行移動することで, 実際に受信した信号\(x(t)\)のスペクトルを正確に再現できる. フーリエ変換における周波数シフト性は次式で表される:

\[
\mathcal{F}\{x(t) e^{-j2\pi f_c t}\}(f) = X(f + f_c)
\]

これにより, 受信信号の特性を正確に分析可能である.


\section{課題2: FM変復調器の原理と考察}

\subsection{FM変調器の原理}
FM変調された信号は次のように表される.

\[
s(t) = A \cos \left( 2 \pi f_c t + k \int m(t) \, dt \right)
\]

ここで,
\begin{itemize}
    \item \( A \): 信号の振幅
    \item \( f_c \): 搬送波の中心周波数
    \item \( k \): 周波数偏移係数
\end{itemize}

FM変調の実装においては, 積分器を使用して信号を処理し, 搬送波の位相を制御する. ブロックダイアグラム(Figure \ref{fig:fm_sender_block})では, 積分器, 変調器, およびUSRPが示されており, 音声信号がFM信号として変調される様子を確認できる.

\begin{figure}[H]
    \centering
    \includegraphics[width=0.8\linewidth]{./data1/day2/fm_sender_block.png}
    \caption{FM送信機ブロックダイアグラム}
    \label{fig:fm_sender_block}
\end{figure}

\subsection{FM復調器の原理}
FM復調は, FM変調された信号 \( s(t) \) の瞬時周波数を検出することで情報信号 \( m(t) \) を復元する方式である. 瞬時周波数は次の式で計算される.

\[
f_{\text{inst}}(t) = \frac{1}{2\pi} \frac{d}{dt} \arg(s(t))
\]

USRPを用いた実装では, 複素IQ信号として搬送波が表現されているため, 次の手順を用いて復調を行う.
1. \( s(t) \) の複素信号の位相を計算する.
2. 位相の微分を求めることで瞬時周波数を得る.
3. 瞬時周波数から情報信号 \( m(t) \) を復元する.

ブロックダイアグラム(Figure \ref{fig:fm_receiver_block})には, 復調プロセスの主要コンポーネントが示されている.

\begin{figure}[H]
    \centering
    \includegraphics[width=0.8\linewidth]{./data1/day2/fm_receiver_block.png}
    \caption{FM受信機ブロックダイアグラム}
    \label{fig:fm_receiver_block}
\end{figure}

\subsection{受信についての考察}
理論的には, IQレートは信号の帯域幅の2倍であるナイキスト周波数を超えていれば十分である. FMラジオでは帯域幅が約200kHzであるため, 最低限のIQレートは400kHzとなる. しかし, 実際にはノイズやフィルタリングの影響を考慮し, 余裕を持たせたIQレートが必要である.

実験では, IQレートを195.312kHz以下に設定することができなかった. これは, USRPが対応する最低サンプリングレートによる制約であると考えられる.

IQレートが低い場合, 帯域幅が狭まり高周波成分が失われるため, 音質が劣化する. しかし, FMの帯域幅が200kHzであるため, IQレートが195.312kHzでも音質の低下は感じられなかった. 仮にIQレートをこれ以上下げることができた場合, 音質の低下が感じられる可能性がある. 実際, 人間の聞こえる周波数帯域は20Hzから20kHz程度である. よって, IQレートを40kHz程度にすると音質の低下が明確になると予想される.

また, 人の会話や音楽の主要な周波数成分はおおよそ1kHz程度であるため, IQレートを2kHzほどに設定すると声や音楽が正しく復元されず, 聴き取れなくなると予想される.

\subsection{送信についての考察}

今回の実験では周波数偏移係数 \(k\) の値を \(10^6\) にした.  
この値を選定する理由について考察を行う.

音声信号を \(m(t) = \sin(2\pi f t)\) とすると,その積分は次式で与えられる:
\[
\int m(t) \, dt = -\frac{\cos(2\pi f t)}{2\pi f}.
\]
このとき,積分結果の振れ幅(最大値)は次のように表せる:
\[
\text{振れ幅} = \frac{1}{2\pi f}.
\]
仮に音声信号の周波数が \(f = 1\,\mathrm{kHz}\) とすると,振れ幅は次のようになる:
\[
\text{振れ幅} = \frac{1}{2\pi \cdot 1000} \approx 0.000159\,\text{(約0.16m)}.
\]

この値は搬送波周波数 \(f_c = 80\,\mathrm{MHz}\) に対して極めて小さい.

FM変調では,この積分結果に周波数偏移係数 \(k\) を掛けて搬送波周波数を偏移させる.したがって, \(k\) の値が十分に大きくない場合,積分値による周波数変化が搬送波周波数に対して目立たなくなり,適切な変調が行えなくなる.

今回の実験では \(k = 10^6\) と設定した.このとき,積分値に基づく周波数変化の振れ幅は次式で示される:
\[
k \cdot \int m(t) dt = 10^6 \cdot 0.000159 \approx 159\,\mathrm{kHz}.
\]
この結果,FMラジオの帯域幅(通常 \(\pm 75\,\mathrm{kHz}\))を十分にカバーするだけでなく,搬送波周波数 \(f_c = 80\,\mathrm{MHz}\) に対して適切な周波数偏移が付与されることがわかる.

以上より,周波数偏移係数 \(k = 10^6\) の設定は,音声信号が適切に変調されるための妥当な選択であると結論付けられる.

\subsection{FMとAMの音質の比較}
FMは, 振幅ではなく周波数で情報を伝達するため, 振幅の変動によるノイズの影響を受けにくい. これにより, AMよりも音質が良くなる. また, FMでは広い帯域幅を利用できるため, 高周波成分を含む信号を忠実に再現することができる. この特性により, FMは音楽や音声の伝送において高い品質を実現できる.

\section{課題3: FSK変復調器の原理と考察}
\subsection{FSK変調器の原理}
FSK(Frequency Shift Keying)はデジタル変調方式の一種であり, ビットの値によって搬送波周波数が変化する. 具体的には, ビット「0」と「1」に異なる周波数 \(f_0\) と \(f_1\) を対応させることで情報を伝送する.

FSK変調された信号は次式で表される:
\[
s(t) = A \cos \left( 2\pi f t \right),
\]
ここで周波数 \(f\) は以下の条件に基づいて選択される:
\[
f = 
\begin{cases} 
f_c + \Delta f & \text{(ビットが1の場合)}, \\
f_c & \text{(ビットが0の場合)}.
\end{cases}
\]
ここで,
\(A\) は信号の振幅, \(f_c\) はキャリア周波数, \(\Delta f\) は周波数偏移である.

この条件に基づいて変調信号は次の式で表される:
\[
v_{\text{FSK}}(t) = A \cos \left[ 2\pi f_c t + 2\pi \Delta f \int_{-\infty}^{t} m(\tau) d\tau \right].
\]
ここで \(m(t)\) は送信したいデジタルデータを表す. この変調プロセス全体の流れは, Figure\ref{fig:fsk_sender_block}に示すブロックダイアグラムに対応する.

\begin{figure}[H]
    \centering
    \includegraphics[width=0.8\linewidth]{./data1/day3/fsk_sender_block.png}
    \caption{FSK送信機ブロックダイアグラム}
    \label{fig:fsk_sender_block}
\end{figure}

Figure\ref{fig:fsk_sender_block}では, デジタルデータの各ビットに対応する周波数変化が, \(n\)(各ビットの時間)の繰り返し処理により実現される. \(n\) はIQレートをシンボルレートで割った値であり, 各ビットの持続時間を規定する. 図下部のforループはこの処理を実現している.

さらに信号を積分をしたのちに, それに \(2\pi\Delta f\) を掛けることで最終的な周波数変調が施される.

\subsection{FSK復調器の原理}
FSK信号の復調器ではまずIQ信号から位相 \(\theta(t)\) を取り出し, これを微分して被変調信号\(2\pi \Delta f m(t)\)を得る. 

微分後の信号から, 周波数偏移量 \(2\pi\Delta f\) を正規化し, コンパレータを用いて信号を「0」と「1」に判定する. Figure\ref{fig:fsk_receiver_block}はこの処理の流れを示したものである.

\begin{figure}[H]
    \centering
    \includegraphics[width=0.8\linewidth]{./data1/day3/fsk_receiver_block.png}
    \caption{FSK受信機ブロックダイアグラム}
    \label{fig:fsk_receiver_block}
\end{figure}

\section{課題4: ASKとPSKの変復調の実装と考察}
\subsection{ASK変調と復調の原理}
ASK(Amplitude Shift Keying)はデジタル変調方式の一種であり, デジタル信号の値に応じて搬送波の振幅を変化させる方式である. ASK変調された信号は次式で表される:
\[
s_{\text{ASK}}(t) =
\begin{cases}
A \cos(2\pi f_c t) & \text{(ビットが1の場合)}, \\
0 & \text{(ビットが0の場合)}.
\end{cases}
\]
ここで, \(A\) は搬送波の振幅, \(f_c\) は搬送波周波数を表す.

ASK変調器では, デジタル信号 \(m(t)\) を搬送波と乗算することで振幅変調を実現する. ASK送信機のブロックダイアグラムのスクリーンショットを撮り忘れたため, ここでは省略する.

ASK信号を復調する際は, 受信信号をエンベロープ検出器に通すことで信号の振幅を復元し, コンパレータを用いて「0」と「1」を判定する. Figure \ref{fig:ask_receiver_block} にASK受信機のブロックダイアグラムを示す.

\begin{figure}[H]
    \centering
    \includegraphics[width=0.8\linewidth]{./data1/day4/ASK_receiver_block.png}
    \caption{ASK受信機ブロックダイアグラム}
    \label{fig:ask_receiver_block}
\end{figure}

受け取った信号を絶対値として取得し, ローパスフィルタを通すことで高周波成分を除去する. さらに, 一定の閾値を超えた場合は1と判定し, それ以外の場合は0と判定する.

Figure \ref{fig:ask_receiver_wave} には, ローパスフィルタ通過前の波形と, 最終的な復調波形を示している.

\begin{figure}[H]
    \centering
    \includegraphics[width=0.8\linewidth]{./data1/day4/ASK_receiver_wave.png}
    \caption{ASK復調後の波形}
    \label{fig:ask_receiver_wave}
\end{figure}

\subsection{PSK変調と復調の原理}
PSK(Phase Shift Keying)はデジタル変調方式の一種であり, ビットの値に応じて搬送波の位相を変化させる方式である. PSK変調された信号は次式で表される:
\[
s_{\text{PSK}}(t) = A \cos(2\pi f_c t + \phi),
\]
ここで, 位相 \(\phi\) は次のように設定される:
\[
\phi =
\begin{cases}
0 & \text{(ビットが1の場合)}, \\
\pi & \text{(ビットが0の場合)}.
\end{cases}
\]

PSK変調器では, デジタル信号に応じて搬送波の位相を変化させることで信号が生成される. Figure \ref{fig:psk_sender_block} にPSK送信機のブロックダイアグラムを示す.

\begin{figure}[H]
    \centering
    \includegraphics[width=0.8\linewidth]{./data1/day4/PSK_sender_block.png}
    \caption{PSK送信機ブロックダイアグラム}
    \label{fig:psk_sender_block}
\end{figure}

途中まではFMと同様の処理をしており, 最後に\(\pi\)をかけることで, 1の信号では位相が反転するようになっている.

PSK信号の復調にはIQ信号から位相を取り出せばよい. Figure \ref{fig:psk_receiver_block} にPSK受信機のブロックダイアグラムを示す.

\begin{figure}[H]
    \centering
    \includegraphics[width=0.8\linewidth]{./data1/day4/PSK_receiver_block.png}
    \caption{PSK受信機ブロックダイアグラム}
    \label{fig:psk_receiver_block}
\end{figure}

Figure \ref{fig:psk_receiver_wave} には, 復調後の信号波形を示している. 位相の違いに基づいて信号が正確に復元されていることが確認できる.

\begin{figure}[H]
    \centering
    \includegraphics[width=0.8\linewidth]{./data1/day4/PSK_receiver_wave.png}
    \caption{PSK復調後の波形}
    \label{fig:psk_receiver_wave}
\end{figure}

\subsection{考察}
ASKとPSKの比較を行う. ASKは信号の振幅を変化させることでデータを送信するため, ノイズに対する耐性が低く, 受信品質が劣化しやすいという特性がある. 一方, PSKは信号の位相を変化させることでデータを送信するため, 振幅の変化による影響を受けにくく, ノイズ耐性が高い.

今回の実験では, Figure \ref{fig:ask_receiver_wave} および Figure \ref{fig:psk_receiver_wave} に示されるように, 両方式とも信号の復元に成功した. しかし, 実環境におけるノイズの影響を考慮すると, PSKがより信頼性の高い方式であるといえる.


\end{document}
