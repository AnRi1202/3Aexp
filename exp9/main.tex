\documentclass[a4paper, twocolumn]{article} % A4用紙, 二段組
\usepackage[utf8]{inputenc} % UTF-8エンコーディング
\usepackage{amsmath, amssymb} % 数式パッケージ
\usepackage{graphicx} % 画像挿入
\usepackage{hyperref} % ハイパーリンク
\usepackage{geometry} % レイアウト設定
\usepackage{caption} % キャプション調整
\usepackage{float} % 画像配置の制御
\usepackage{titlesec} % セクションの見た目変更
\usepackage{siunitx} % SI単位系
\usepackage{multicol} % 必要に応じて複数列対応

% ページの余白を調整
\geometry{
  top=25mm,
  bottom=25mm,
  left=20mm,
  right=20mm
}

% タイトルとセクションのフォーマット調整
\titleformat{\section}{\large\bfseries}{\thesection}{1em}{} % セクションタイトル
\titleformat{\subsection}{\normalsize\bfseries}{\thesubsection}{1em}{} % サブセクションタイトル

% タイトル情報
\title{後期実験9}
\author{学籍番号: 03240470 \and 氏名: 井手陸大}
\date{\today}

\begin{document}

% タイトル
\maketitle
\begin{abstract}
本レポートでは、実験の目的、方法、結果、考察について記述する。以下に内容の概要を示す。
\end{abstract}

% セクション1: 実験の目的
\section{実験の目的}
この実験の目的は、○○の物理特性を測定し、その挙動を理解することである。

% セクション2: 実験方法
\section{実験方法}
\subsection{使用した機器}
\begin{itemize}
    \item オシロスコープ
    \item マルチメーター
    \item 電源装置
\end{itemize}

\subsection{手順}
以下の手順で実験を行った。
\begin{enumerate}
    \item 回路を組み立てた。
    \item 各測定点の電圧を測定した。
    \item データを記録し、結果を整理した。
\end{enumerate}

% セクション3: 実験結果
\section{実験結果}
\subsection{測定データ}
表\ref{tab:results}に測定したデータを示す。

\begin{table}[H]
    \centering
    \caption{測定結果}
    \label{tab:results}
    \begin{tabular}{|c|c|c|}
        \hline
        測定点 & 電圧[V] & 電流[A] \\
        \hline
        A      & 5.0     & 0.1     \\
        B      & 10.0    & 0.2     \\
        C      & 15.0    & 0.3     \\
        \hline
    \end{tabular}
\end{table}

\subsection{グラフ}


% セクション4: 考察
\section{考察}
今回の実験結果から、以下のことが分かった。
\begin{itemize}
    \item 測定値は理論値とほぼ一致していた。
    \item 誤差の原因として、接触抵抗の影響が考えられる。
\end{itemize}

% セクション5: 結論
\section{結論}
今回の実験を通して、○○の物理特性について深く理解することができた。

% 参考文献
\section*{参考文献}
\begin{enumerate}
    \item 著者名, 「タイトル」, 出版社, 出版年.
    \item John Doe, "Physics for Engineers", Springer, 2020.
\end{enumerate}

\end{document}
